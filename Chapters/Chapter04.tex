\chapter{Quantitative dark-field imaging}\label{ch:lung-dark-field}
\section{Introduction}\label{sec:introduction}
X-ray grating interferometry has been developed over the course of the last
  fifteen years on both synchrotron and laboratory
sources~\cite{David_2002,1347-4065-42-7B-L866,Weitkamp_2005,1347-4065-45-6R-5254,Pfeiffer2006}. Interferometric
imaging allows simultaneous access to three independent images: the
conventional absorption image, a differential phase signal and a dark-field
signal, also known as visibility contrast. This last signal has been
reported by multiple sources as being quantitatively linked to the presence
of unresolved structures in the sample, much smaller than the pixel size of
the detector, and typically of the order of the
micrometer~\cite{Pfeiffer2008,Lynch:11,Yashiro:10}.
Various clinically relevant
applications~\cite{Wen_2009,Thilo2013} have been
proposed, including lung microstructural analyses~\cite{Schleede17880,Meinel_2014,Meinel_2013,Yaroshenko_2013}.

In particular, emphysema is a pathological condition of the lung
resulting in structural changes in the alveoli, that is at the smallest
hierarchical level in the tissue. These changes are caused by the
destruction of interalveolar septa, with larger air spaces progressively
replacing the fine architecture of the lung parenchyma~\cite{Sharafkhaneh_2008}. These larger spaces
have a less favorable surface-to-volume ratio, thus lowering the efficiency
of respiratory exchanges. Absorption radiography proved to be accurate in
diagnosing emphysema only in the advanced stages of the disease. 
High-resolution computed tomography is able to detect regions in the lung
with abnormally low attenuation, at the cost of exposing the patient
to a higher radiation dose.

Previous studies on murine lungs established that the increased sensitivity of
the dark-field signal of an X-ray grating interferometer to micrometer-sized
features can distinguish between emphysematous from healthy samples and
provide a mapping of the parenchyma showing the localization of the
structural damage.
The strength of the dark-field signal in grating interferometry is
given by the small-angle scattering of X-rays by structures smaller than the
spatial resolution of the imaging system. The lung is therefore an ideal
application for this technique, since the alveoli are much
smaller than the spatial resolution available for chest radiography.

In this work we present a quantitative model of the dark-field signal
generated by lung tissue in an X-ray grating interferometer on a laboratory
source. High-resolution tomographic data resolving the features down to single
alveoli is analyzed with established post-processing techniques to extract a
ground truth on the sizes of the structures composing the lung tissue. The
lung is then modeled as a suspension of spheres of different sizes in a
homogeneous medium, in the hypothesis that the signals generated by spheres
of different sizes and by X-rays of different energies sum up incoherently.
Finally, an X-ray interferometric radiography on a laboratory source is
recorded, to allow a direct comparison between the expected dark-field
signal calculated according to this model and the experimental values.


\section{Methods}\label{sec:methods}
\subsection{Sample preparation}
Eight \emph{ex-vivo} mouse samples were prepared at the university of Bern.
In order to preserve the delicate structure of the lungs, critical point
drying is used in order to avoid the damage caused by evaporating the liquids
from the sample in ordinary pressure conditions.

\subsection{Image acquisition}\label{sec:acquisition}
The microtomography high-resolution 3D images were acquired at the X02DA
TOMCAT beamline of the Swiss Light Source at the Paul Scherrer Institute
(Villigen, Switzerland). The X-ray beam is generated with a 2.9 T bending
magnet from electrons at an energy of 2.4 GeV. The current in the storage
ring is 400 mA, top-up mode. Two multilayer monochromator crystals are used
to filter X-rays with an energy of 11 keV. A 20 \selectlanguage{greek}μ\selectlanguage{english}m thick scintillator
converts the X-rays into visible light, collected by a 10x objective onto a
high-speed CMOS sensor with an effective pixel size of $0.65 \times 0.65$
\selectlanguage{greek}μ\selectlanguage{english}m$^2$. The exposure time for each tomographic scan was set at 100 ms per
projection, with 1801 projections.

The radiographies on the laboratory source were taken on a symmetric
Talbot-Lau interferometer (see fig.~\ref{248327}) with a grating pitch of 5.4 \selectlanguage{greek}μ\selectlanguage{english}m and an intergrating
distance of 26 cm, the phase-shift grating provides a phase shift of $\pi$
at 45 keV. The average visibility of the interference pattern without any
sample is 8\%. The source is a Comet MXR-225/26 X-ray tube operated at 100
kV and 6 mA. The source size is 1 mm. The detector is a prototype based on
Santid CdTe by Dectris Ltd. The CdTe 750 \selectlanguage{greek}μ\selectlanguage{english}m sensor provides high quantum
efficiency at high energies (>90\% at 60 keV) with a pixel size of 75 x 75
\selectlanguage{greek}μ\selectlanguage{english}m$^2$. The phase-stepping procedure was performed with 31 phase steps with
1 s exposure per step.\selectlanguage{english}
\begin{figure}[h!]
\begin{center}
\includegraphics[width=0.70\columnwidth]{figures/lung-setup/lung-setup}
\caption{{Talbot-Lau interferometer with three gratings G0, G1 and G2, used for
the radiographies on the laboratory source.
{\label{248327}}%
}}
\end{center}
\end{figure}


\subsection{Processing of the tomographic data}\label{sec:tomoprocessing}
The tomographic sinograms are preprocessed with the Paganin
algorithm~\cite{Paganin_2002} and
the 3D volume is then reconstructed with the gridrec
algorithm~\cite{Marone:pp5022}. The volumes
are then thresholded with the Otsu algorithm~\cite{Otsu_1979}, with independent thresholds
for each slice, to provide a binary labelling for tissue and air. A cycle of
erosion and dilation is applied to remove single pixel artifacts while still
keeping the septa between neighboring alveoli. This preserves the structures
in the lung because septa, while possibly being only one pixel in thickness,
are also connected to surrounding tissue. On the other hand, isolated pixels
should be removed as they could affect the subsequent analysis.

Next, each lung sample is manually stitched from three or four local
tomographies, depending on the sample thickness, and the local air thickness
map is calculated. The local thickness map is an algorithm developed for
bone analysis or foam analysis in material science~\cite{6778077}, that aims
at calculating the maximum diameter of a sphere fitting in each hole in the
sample.

The euclidean distance transform is calculated from the image, yielding the
distance of each point in the airways from the closest wall. The points in
the transformed dataset are then sorted in descending order in order to fit
the largest possible sphere at each location.

In the case of lungs, each of the airways is filled with the largest possible
sphere, resulting in a model where the lung airways are a collection of
spheres embedded in the background of the tissue.

To obtain a distribution of sphere sizes, kernel density estimation in the R
statistics package \emph{ks}~\cite{JSSv021i07} is used in order to avoid possible artifacts
resulting from the arbitrary binning of the histogram of sphere sizes.

\subsection{Processing of the radiographic data}\label{sec:radioprocessing}
For each pixel, the transmission image $A$ and the
dark-field image $B$ are calculated (see fig.~\ref{590406}). Both of these
quantities are known to follow the Beer-Lambert law, with an exponential
decay related to the thickness $t$ of the sample: $A = \exp(-\mu_A t)$, $B =
\exp(-\mu_B t)$. Since the lung samples are irregular in shape, it is
beneficial to examine the ratio of the logarithms $R$ in
order to remove the local dependence on the thickness of the sample in the
radiography:

\begin{align*}
    A &= \dfrac{a_{0,s}}{a_{0,f}}\\
    B &= \dfrac{a_{1,s}}{a_{1,f}}\dfrac{a_{0,f}}{a_{0,s}}\\
    R &= \dfrac{\log(B)}{\log(A)}.
    \label{eqn:definitions}
\end{align*}\selectlanguage{english}
\begin{figure}[h!]
\begin{center}
\includegraphics[width=0.70\columnwidth]{figures/KO373-LL-smoke/WT256-LL-smoke}
\caption{{Example of an ex vivo lung interferometric radiography on a laboratory
source. Sample labelled WT256.
{\label{590406}}%
}}
\end{center}
\end{figure}

\section{Modelling the polychromatic dark-field signal}\label{sec:model}
Previous works~\cite{Lynch:11,Gkoumas2016} have demonstrated quantitative estimation of
the dark-field signal for a monochromatic source and monodisperse solutions
of micrometer-sized spheres. In particular, the coefficient $\mu_B$, for a
beam of wavelength $\lambda$ is:

\begin{equation}
\mu_B = \frac{3\pi}{\lambda^2}f |\Delta n|^2 d
    \begin{cases}
    D' & \text{if } D' \leqslant 1\\
    \begin{align}
    & D' - \sqrt{D'^2 - 1}\\
    & (1 + D'^{-2}/2) \\
    & + (D'^{-1} + D'^{-3} / 4) \\
    & \log\left(\frac{D' + \sqrt{D'^2 - 1}}{D' - \sqrt{D'^2 - 1}}\right)
    \end{align} & \text{otherwise.}
    \end{cases}
    \label{eqn:lynch}
\end{equation}

The autocorrelation length is $d = L\lambda / p$ with $L$ the sample to
detector distance, $p$ and the period of $G_2$. $D'$ is a normalized
particle diameter equal to $D/d$, where $D$ is the particle diameter. $f$ is
the fraction of volume occupied by the scattering material and $n = 1 -
\delta - i\beta$ is the complex refractive index.

We rewrite this formula more concisely as

\begin{equation}
    \mu_B(\energy) = C |\Delta n(\energy)|^2 \energy u(\energy),
    \label{eqn:lynchshort}
\end{equation}
where $\energy$ is the energy of the beam, $C = 3 fL / 4p$ is a constant
depending only on the setup geometry and the volume fraction $f$ occupied by
the spheres and $\energy$ is the energy of the beam.


In the case of a polychromatic source, we tested a model where different
energies interact with the sample independently of each other, thus allowing
an incoherent sum of the dark-field signals over the spectral weights
$s(\energy)$:

\begin{equation}
    R = C \frac{\sum_\energy s(\energy)|\Delta n(\energy)|^2 \energy u(\energy)}{\sum_\energy s(\energy) \energy \beta}.
    \label{eqn:lynchpolychromatic}
\end{equation}

This formula was first tested on monodisperse solution of silicon dioxide
spheres with sizes ranging from 0.166 \selectlanguage{greek}μ\selectlanguage{english}m to 7.760 \selectlanguage{greek}μ\selectlanguage{english}m, with a volume fraction of 20\%
in glycerine. This model yields a good agreement between the continuous line
and the experimental data representing radiographies of the silicon dioxide
sphere samples recorded on the laboratory source interferometer.\selectlanguage{english}
\begin{figure}[h!]
\begin{center}
\includegraphics[width=0.70\columnwidth]{figures/summary/summary}
\caption{{Silicon dioxide monodisperse microspheres in a 20\% solution of
glicerine. as a function of sphere diameter. The continuous line is the
function in eq. {\ref{eqn:lynchpolychromatic}}.
{\label{725462}}%
}}
\end{center}
\end{figure}


A model of the sample is required to calculate the spectral weights
$s(\energy)$ in order to avoid beam hardening artefacts. The source spectrum is
simulated with the SpekCalc~\cite{spekcalc} software, then attenuated
according to NIST absorption tables~\cite{Hubbell_1995} for the sample in the beam path and the
efficiency of the detector.

In this work, we propose to model the lung sample as a solution of
polydisperse spherical air bubbles embedded in tissue. The distribution of
the sphere diameters is extracted from the microtomographic data, together
with the fraction of the volume occupied by the tissue. The density of the
tissue is also estimated from tomographic projections: the composition is
taken from the ICRU-44 lung tissue tables~\cite{White_1989}, while the density is increased to
1.6 g/cm$^3$ as to match the absorption values recorded on the tomographic
projections. This results from the drying procedure for the fixation of
the samples, leaving a denser material than in \emph{in vivo} conditions.

The coefficient $\mu_B$ can then be expressed as a double summation over the spectrum
$s(\energy)$ and over the distribution of sphere diameters $\rho(d)$:

\begin{equation}
    \mu_{B,\text{total}} = \sum_{d}\sum_{\energy} \mu_B(r, E; n, f)\rho(d)s(\energy)
    \label{eqn:totalsum}.
\end{equation}


\section{Results and discussion}\label{sec:results}
The section of the radiographic image of each lung sample is manually
matched to the local tomography, as identified during the alignment of the
tomographic scan. The average and standard deviation of the $R$ values are
calculated and plotted in fig.~\ref{206272} (black dots and errorbars). The expected
values according to eq.~\ref{eqn:totalsum} are calculated with the inputs from the
microtomographic datasets, as described in section~\ref{sec:tomoprocessing}, and are plotted on
fig.~\ref{206272} with red dots for each sample.\selectlanguage{english}
\begin{figure}[h!]
\begin{center}
\includegraphics[width=0.70\columnwidth]{figures/samples/samples}
\caption{{\(R\) value for each lung in the laboratory setup
radiography (measured, black dots with errorbars) compared to the
expected value modelled from microtomographic inputs (theory).
{\label{206272}}%
}}
\end{center}
\end{figure}


The measured data agreed with the
theory for all samples, and the $\chi^2$ statistic can be calculated

\begin{equation}
    \chi^2 = \sum_{i=1}^8 \dfrac{(R_{\text{obs}} -
    R_{\text{th}})^2}{\mathop{\mathrm{Var}}(R_{\text{obs}})} = 2.31.
    \label{eqn:chisq}
\end{equation}

With 7 degrees of freedom, this results in a right-tail probability of 0.94 that the 
observations are consistent with the model, thus validating the quantitative estimation of
dark-field values for a lung model as a collection of incoherently
superimposed microscopic spheres.

Possible sources of inaccuracy of our model include a significantly
different distribution of shapes in the lung parenchyma, or additional
effects of beam hardening which are known to influence the dark-field
signal. In order to consider the effects of possible shape asymmetries, a
single mouse lung sample was scanned at the X02DA TOMCAT beamline with the
setup described in Kagias et al.~\cite{PhysRevLett.116.093902}. This
technique is similar to Talbot interferometry, but it is able to detect
scattering from unresolved microscopic structures under different angles in
a single shot by
creating an array of circular interference patterns. The intensity of the
dark-field signals for different angles can then be recovered for each
pixel, and an asymmetry value defined as the amplitude of this periodic
signal divided by the average can be calculated for each pixel. This allows
us to exclude the possibility that there are significant inhomogeneities in
the dark-field response of the lung structures under different angles, which
would directly invalidate the model of a superposition of spheres. This
amounts to quantifying the departure from a spherical model towards
ellipsoids. In our sample this asymmetry is an average of $0.11 \pm 0.04$, indicating a departure from the spherical model of less than 15\%.

Another effect commonly reported in dark-field analyses on wide, polychromatic
sources is the influence of beam hardening on the recorded dark-field
values. In our case, given the high voltage of the source and the small
thickness of the samples (less than 4 mm), the absorption ranges from 4\% to
6\%. The correction to dark-field values provided by is therefore
not applied, as it is reported to be relevant for samples absorbing at least
50\% of the incoming light. The beam hardening effect is considered insofar
it affects the spectral weights of eq.~\ref{eqn:totalsum}, which are calculated
after the sample. 
