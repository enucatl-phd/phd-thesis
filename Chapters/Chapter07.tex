\chapter{Conclusions and outlook}\label{ch:conclusions}
The main goal of this work was the realization of Talbot-Lau interferometric
setups that are able to operate in the energy ranges that are typical for
medical diagnostics and material nondestructive testing.

A first design has been realized in the edge-on configuration, where the
gratings are illuminated from one side. This employs the full length of the
grating grooves as the absorbing structure, thus achieving an effective
stopping power for high energy photons, with a transmission below
\SI{1}{\percent} at \SI{100}{\kilo\eV}. Two setups with a design energy
of~\SI{100}{\kilo\eV} and~\SI{120}{\kilo\eV} have been installed and their
imaging capabilities have been demonstrated (chapter~\ref{ch:edgeon}).
Further optimizations include a more efficient detector design, with an
edge-on silicon sensor, shown in figure~\ref{sec:mythen-edge-on}. This
allowed the current experiments to achieve an overall average visibility
around~\SI{7.5}{\percent}. This is still not excellent, but serious
difficulties in the microfabrication techniques have to be taken into
consideration, and improvements are to be expected considering the ongoing
research in the field. The setups could establish preliminary quantitative
measurements of phase contrast intensities of known materials and
qualitative assessments of dark-field imaging for samples with unresolved
microstructures with sizes below the pixel size of the detector
(chapter~\ref{ch:towards}).

The second aim was the operation of a Talbot-Lau interferometer at energies
still relevant for medical examinations but on a two-dimensional
configuration. The goal was set for the design of a setup with a tube
voltage of~\SI{120}{\kilo\eV}. Relaxed goals for grating fabrication
included a minimum absorption thickness of~\SI{100}{\micro\meter} of gold,
with a pitch of~\SI{5.4}{\micro\meter}. This aimed at achieving a higher
quality in the optical elements by not pushing for extreme sensitivity
requirements of the interferometer itself.
Silicon photon counting detectors are shown
to be inadequate in this energy range and cadmium telluride prototypes could
be successfully tested for general purpose imaging and for the development
of a quantitative model for mapping the dark-field signal in an
interferometric radiography to the size of alveoli in mouse lungs
(chapter~\ref{ch:lung-dark-field}). This is a
potentially relevant tool for the early detection of lung pathological
conditions. This is currently not achievable by absorption radiography but
would be critical for the assessment of degenerative conditions such as
emphysema whose progression can be slowed with a timely intervention.

Finally, a setup that goes beyond Talbot-Lau interferometry has been
realized. The omnidirectional technique improves the one-dimensional
sensitivity of Talbot-Lau interferometry by employing gratings with circular
patterns. Talbot-Lau interferometry can detect differential phase and
dark-field signals related to displacements of the wave front in the
direction perpendicular to the grating lines. An omnidirectional circular
setup is sensitive to differential phase and dark-field in all directions
and it is able to provide a directional mapping for the
dominant alignment of fibers in a sample, as shown in
chapter~\ref{ch:omnidirectional}.

The potential for X-ray phase-contrast imaging at high energies, suitable
for general purpose examinations in a wide variety of fields has been
qualitatively and quantitatively confirmed. The realized setups are suitable
for an extension to larger fields of view and a better performance,
depending on the critical step involving the fabrication of the optical
elements.
Alternative approaches to conventional Talbot-Lau interferometry could also be extended
from synchrotron or microfocus sources to sources with a lower degree of
spatial coherence. Omnidirectional interferometry is an example, and it also
has an influence on the availability of
the required optical elements. An omnidirectional 
interferometer does not require an absorption analyzer grating like Talbot
interferometry. Establishing this method on low brilliance laboratory
sources removes a significant hurdle in the fabrication of absorption
gratings. Other techniques could also be demonstrated on high-energy
laboratory sources, such as \emph{dual-phase} interferometry. Dual-phase
interferometry uses the beat frequency between two beam splitter gratings in order to create an
interference pattern with a large period that can be directly resolved by
the detector. This would not only make the fabrication of optical elements
much easier, but also provide the instrument with additional flexibility
concerning the sensitivity of the dark-field signal to different structure
sizes, simply by tuning the geometry of the system.
